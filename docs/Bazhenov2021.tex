\documentclass[12pt, twoside]{article}
\usepackage{jmlda}
\newcommand{\hdir}{.}

\begin{document}

\title
    {Поиск границ радужки методом круговых проекций} % окончательное ли название?
\author
    {А.\,А.~Баженов, И.\,А.~Матвеев} % основной список авторов, выводимый в оглавление
\email
    {bazhenov.aa@phystech.edu; ivanmatveev@mail.ru}
%\thanks
%    {Работа выполнена при
%     %частичной
%     финансовой поддержке РФФИ, проекты \No\ \No 00-00-00000 и 00-00-00001.}
%\organization
%    {$^1$Организация, адрес; $^2$Организация, адрес}
\abstract
    {В работе рассматривается задача сегментирования изображения глаза. Входными данными являются изображение и считающееся известным положение зрачка глаза. Для нахождения границ зрачка и радужки используется нейронная сеть, для достижения максимальной производительности алгоритма используется предварительная обработка данных. Работа алгоритма проверена на базе изображений.
	
\bigskip
\noindent
\textbf{Ключевые слова}: \emph {}

}

%данные поля заполняются редакцией журнала
%\doi{10.21469/22233792}
%\receivedRus{01.01.2017}
%\receivedEng{January 01, 2017}

\maketitle
\linenumbers

\section{Введение}
Сегментация изображения глаза человека является одним из важнейших этапов идентификации личности. В обзоре [1] приведено несколько десятков статей, описывающих использование радужки в биометрических целях. В статье [2] описана общая схема работы системы сегментации изображения глаза: нахождение приблизительной позиции зрачка и последующее нахождение границ зрачка и радужки, с возможным итеративным уточнением.

В [2, 3] для реализации этапа первоначального определения границ радужки  используется метод круговых проекций. Круговая проекция яркости~--- интеграл градиента яркости изображения по окружности, имеющей центр в предполагаемом центре зрачка, либо по ее дуге. По предположению из [3], найдя точку локального максимума зависимости круговой проекции яркости от радиуса окружности, можно найти радиус границы радужки. Однако на яркость изображения в районе границы может оказываться влияние затемнения от ресниц и других элементов лица, что делает возможность эвристических алгоритмов, используемых в [2, 3] ограниченным.

Целью работы является исследование методов, которые возможно использовать для обработки результатов подсчета круговых проекций, причем более устойчивых к влиянию внешних факторов, чем эвристические алгоритмы. Один из таких методов~--- использование нейронной сети. Именно этом метод было решено исследовать в рамках работы.


%%%% если имеется doi цитируемого источника, необходимо его указать, см. пример в \bibitem{article}
%%%% DOI публикации, зарегистрированной в системе Crossref, можно получить по адресу http://www.crossref.org/guestquery/
\begin{thebibliography}{99}


 \bibitem{article}
    \BibAuthor{A.~Nithya, C.~Lakshmi}
   Iris Recognition Techniques: A Literature Survey~//
    \BibJournal{International Journal of Applied Engineering Research}, 2015
	
 \bibitem{article} 
    \BibAuthor{K.\,A.~Gankin, A.\,N.~Gneushev, and I.\,A.~Matveev}
   Iris image segmentation based on approximate methods
with subsequent refinements~//
    \BibJournal{Journal of Computer and Systems Sciences International}, 2014. Vol.~53. \No\,2. pp.~224--238.
	\BibDoi{10.1134/S1064230714020099}.
	
  \bibitem{article}
    \BibAuthor{I.\,A.~Matveev}
   Detection of iris in image by interrelated maxima of brightness gradient projections~//
    \BibJournal{Appl. Comput. Math.}, 2010. Vol.~9. \No\,2. pp.~252--257.
	
 
 	
\end{thebibliography}

%%%% если имеется doi цитируемого источника, необходимо его указать, см. пример в \bibitem{article}
%%%% DOI публикации, зарегистрированной в системе Crossref, можно получить по адресу http://www.crossref.org/guestquery/.

\end{document}
